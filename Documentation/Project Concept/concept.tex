\documentclass[pdftex,12pt,letter]{article}
\usepackage{fancyhdr}
\usepackage{enumerate}
\usepackage{tabularx}
\usepackage{graphicx}
\usepackage{array}
\usepackage[toc,page]{appendix}
\usepackage[justification=justified,singlelinecheck=false]{caption}
\usepackage{placeins}
\pagestyle{fancy}
\makeatletter
  \renewcommand\@seccntformat[1]{\csname the#1\endcsname.\quad}
\makeatother

\newcolumntype {Y}{ >{\raggedright \arraybackslash }X}
\newcommand{\HRule}{\rule{\linewidth}{0.5mm}}
\captionsetup{labelformat=empty}

\begin{document}

\begin{titlepage}
\begin{flushright}
\HRule \\[0.4cm]
{ \bfseries
{\huge EECS 395 Project Concept\\[1cm]}
{\Large for\\[1cm]}
{\huge Chocolate Box\large\\[.1cm]
A Procedural Level Generator for Unity\\[3cm]}
{\large Prepared by\\[1cm]James Fitzpatrick\\Stuart Long\\Frank Singel\\[2cm]
Version 1.0\\
January 27, 2014\\
}}
\end{flushright}
\end{titlepage}
\begin{table}[!h]
\caption*{\bfseries Revision History}
\begin{tabularx}{\textwidth }[t]{|l|l|Y|l|}
\hline
\bfseries Name & \bfseries Date & \bfseries Reasons for Change & \bfseries Version \\ \hline
Long, Fitzpatrick, Singel & 1/17/2014 & Initial Draft & 1.0\\\hline
\end{tabularx}
\end{table}
\FloatBarrier
\newpage

\section{Introduction}
\textit{Chocolate Box} will be a library for Unity 4.3. Unity is one of the most popular, free game engines of today and the newest version of Unity introduced an entirely new 2D framework. It is towards this framework that this library will be targeted. In short, \textit{Chocolate Box} will be a procedural level generator for any general 2D game. Generating a level procedurally entails programmatically building a pseudo-random game level that follows certain constraints. This manner of design allows games to have an infinite variety of levels for their players. Any 2D game developer will simply specify certain parameters to the library and then the library will automatically generate a new, unique level that blends smoothly into the developers gameworld. This generation means that no two playthroughs of a game are ever quite the same. Level generation of this sort can be seen in games such as \textit{Starbound}, \textit{Spelunky}, and \textit{Minecraft}. \textit{Chocolate Box} will be intended for platformer games but could have applications in games of other genres as well. The problem of procedural level generation is a challenging one due to the concerns of ensuring the generated levels is a fun, playable game that constrains to a developer's parameters.
\section{Features}
\begin{description}\itemsep1pt
\item[Procedurally generate levels:] The library will generate levels programatically based on user inputs.
\item[User parameters:] Users will be able to specify a number of parameters in order to adjust what kind of level is generated. These parameters will include items such as complexity, length, difficulty, and size.
\item[User defined assets] The generated levels will be able to use user assets for texturing. The library will provide a convenient way for the user to manage any assets he sends to the library.
\item[Enemy Spawning:] The level generator will automatically spawn enemies provided by the user.
\item[Non-environment game objects] The level generator will be able to incorporate any various game objects the user may wish to include. Such objects could be checkpoints, powerups, or score items.
\item[Varied Environment] The environment or terrain generated by the level will be very varied and able to encorporate many traditional platforming features. These feature include platforms, hostile terrain such as pits, a background, and varied decorative items.
\item[Boss area] Because \textit{Chocolate Box} is intended for platformer games, the user will be able to specify whether or not a specific boss area should be included at the end of the generated level. The actual boss and boss area constraints will be provided by the user.
\end{description}

\section{Challenges}
\begin{description}\itemsep1pt
\item[Playability] Any levels generated must be fully playable within the constraints of the current gameworld.
\item[Incorporating User Assets] Unknown user assets must be successfully incorporated.
\item[Accommodating User Parameters] The generated level must have the characteristics reflecting the users set parameters.
\item[Level randomness] Each level must achieving the correct balance between randomness and scriptedness in the level.
\item[Uniqueness] Any given generated level must be identifiably different from other generations.
\item[Flow] Each generated level must flow as desired by the game creator.
\item[Performance] There must not be an unreasonable performance issue on level load.
\end{description}

\section{Team Responsibilities}
\begin{table}[!h]
\begin{tabularx}{\textwidth }[t]{|l|Y|}
\hline
\bfseries Team Member & \bfseries Responsibilities\\\hline
James Fitzpatrick & Any non-environmental objects including enemies.\\\hline
Stuart Long & Project manager and all features related to environmental objects.\\\hline
Frank Singel & User interface, user parameters, and the demo game.\\\hline
\end{tabularx}
\end{table}
\end{document}